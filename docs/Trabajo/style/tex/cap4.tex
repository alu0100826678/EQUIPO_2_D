%%%%%%%%%%%%%%%%%%%%%%%%%%%%%%%%%%%%%%%%%%%%%%%%%%%%%%%%%%%%%%%%%%%%%%%%%%%%%
% Chapter 4: Conclusiones y Trabajos Futuros 
%%%%%%%%%%%%%%%%%%%%%%%%%%%%%%%%%%%%%%%%%%%%%%%%%%%%%%%%%%%%%%%%%%%%%%%%%%%%%%%

Mediante la realizaci�n del trabajo hemos contrastado la informaci�n con programas inform�ticos (Python), por lo cual,
queda demostrado la eficacia frente a otros m�todos, ya sea por la facilidad que aporta al hacer investigaciones 
o por su precici�n de c�lculo. Con esta herramienta de trabajo hemos concluido que el Desarrollo de Taylor aplicado a la 
funci�n logaritmo neperiano de ``x'', fijando dos varibles y variando una, afecta, sobretodo, a la proximidad de la Serie
de Taylor en f(x)=ln(x) y el valor del Logaritmo Neperiano en ese punto. Los resultados de la parte experimental, respaldados por la parte te�rica 
en la que se demuestra como hallar el polinomio de Taylor del ln(x) mediante un sumartorio general o un sumatorio solo aplicable al logaritmo 
neperiano, nos indican que el error ser� minimo cuanto mayor sea el grado o nulo si el punto y el centro a tratar son iguales; tambi�n,
hemos podido observar el efecto contrario al variar el centro.
Adem�s, hemos empleado los procesadores de texto LaTeX y Beamer, lo que nos ha llevado a ampliar nuestros conocimientos en esta materia.
