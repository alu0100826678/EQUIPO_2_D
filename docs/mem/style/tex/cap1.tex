%%%%%%%%%%%%%%%%%%%%%%%%%%%%%%%%%%%%%%%%%%%%%%%%%%%%%%%%%%%%%%%%%%%%%%%%%%%%%
% Chapter 1: Motivaci�n y Objetivos 
%%%%%%%%%%%%%%%%%%%%%%%%%%%%%%%%%%%%%%%%%%%%%%%%%%%%%%%%%%%%%%%%%%%%%%%%%%%%%%%

El objetivo de esta pr�ctica es demostrar los conocimientos adquiridos en Latex, Beamer y Python en el transcurso de la asignatura de T�cnicas Experimentales.
Aplicaremos los programas ya mencionados en la realizaci�n de un informe sobre la funci�n logaritmo neperiano y su desarrollo de Taylor.
Adem�s implementaremos un programa en Python que nos ayudar� a realizar varios experimentos para respaldar nuestras afirmaciones sobre el tema planteado.
%----------------------------------------------------------------------------------------------------
\section{Uso de Python,Beamer y Latex}
\label{1:sec:1}
\begin{itemize}
\item \LaTeX{} : Utilizaremos este programa en la realizaci�n del informe que presentaremos sobre $f(x)=Ln(x)$
\item Beamer : Recurriremos a la creaci�n de diapositivas para orientarnos en la exposici�n oral del trabajo.
\item Python : Crearemos un programa en lenguaje interpretado Python que utilizar� el desarrollo de Taylor para aproximar la funci�n Ln(x). 
As� como calcularemos el tiempo que tarda en realizar dicha aproximaci�n.
\end{itemize}
%---------------------------------------------------------------------------------




